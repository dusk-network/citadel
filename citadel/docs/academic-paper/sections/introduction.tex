
The amount of services available on the Internet is increasing year after year, as well as the concerns about how Service Providers (SPs) handle the sensitive information about their users. In this scenario, SPs are entities that users need to trust, especially regarding the fact that they will behave correctly when managing their personal information. In this regard, Self-Sovereign Identity (SSI) systems \cite{sovrin, sov1, belles2022selfsovereign} have become a research area of interest in the last few years. When it comes to building privacy-preserving authentication methods in services, they play a prominent role: grant users a way to manage their identities in a fully transparent manner. In other words, users of a service implementing an SSI system, are aware of all the sensitive information the SP is requesting, and they can consent to or deny each request.

SSI systems are useful in a wide set of scenarios, like distributing tickets for events, managing users of video or music streaming subscriptions, using car or parking sharing applications, etc. But these examples are not the only kind of contexts where SSI systems are an important privacy feature to implement. If we move our attention to the Internet of Things (IoT) paradigm, we observe a growing density of IoT devices in our cities: medical devices, pollution sensors, traffic lights, IP cameras, etc. In some examples, like an autonomous car sharing its location with an SP or other cars nearby, being able to share only such information but preventing traceability at the very same time, would be an interesting feature to implement. In other words, would be interesting to have SSI systems in these scenarios as well.

\subsection{Related Work} 

Interesting SSI approaches have been introduced recently, like the one detailed in \cite{DBLP:journals/corr/abs-2007-00415}. This paper states a way to deploy an SSI system that grants anonymity to its users at the network level. Other solutions like \cite{abid2022novidchain} introduce a Blockchain-based system for preserving the privacy of users when managing their vaccine certificates.

In most cases, state-of-the-art SSI systems use Zero-Knowledge Proofs (ZKPs) \cite{gmr85} as the backbone of their architecture: cryptographic primitives allowing users to prove knowledge of some information, without leaking anything about it. This is the case of \verb!SANS! \cite{salleras2020sans}, where the authors introduce a private authentication protocol based on these primitives. Using \verb!SANS!, users can prove their rights to access different services, without the SP knowing the identity of the users.

Often, they also rely on Blockchain technologies \cite{blockchain2}, in order to achieve decentralization and immutability, when it comes to buying and granting rights to users. For instance, projects like \textit{iden3}\footnote{https://iden3.io} or \textit{Jolocom}\footnote{https://jolocom.io} build SSI systems where owners of Decentralized Identities (DIDs) are able to manage them in a private manner. At the time of writing this, both solutions rely on the Ethereum Blockchain.

On the other hand, \verb!FORT! \cite{math10040617} is an SSI system that relies on Non-Fungible Tokens (NFTs): assets uniquely identifiable that contain some specific information. What they do, is represent the right acquired by someone as an NFT stored on a Blockchain, and they can prove ownership of this right by means of a ZKP. However, even when it does a great job preserving the privacy of the users of different services, this solution still has some open problems to address: the NFTs, as implemented nowadays, are publicly stored on Blockchains like Ethereum. This means that, even when users can privately prove ownership of such rights, they can still be traced on-chain. As stated in the open problems section of their paper, the authors explain how being able to integrate their solution into Blockchains like Dusk Network would lead to enhanced privacy. Dusk Network is a Blockchain where all the transactions are private by default, and capable of executing smart contracts with built-in privacy features. Being able to integrate a private-by-design NFT model into Dusk, would lead to the possibility of designing and deploying an SSI system on top of it, which would prevent on-chain traceability.

Furthermore, \verb!FORT! presents another problem: the SPs need to be trusted, as the ZKPs sent to them could be reused by them, impersonating like this the users. As such, finding a way to ensure that a license that has been already used cannot be reused in other scenarios, would be a desirable feature.

\subsection{Contributions} 

In this paper, we introduce two main contributions. First, we design a private NFT model to be integrated into Dusk Network. Using such a model, a user buying an NFT will receive a token that only they will be able to read. This approach has full integration with the Dusk Network Blockchain: the changes to the original protocol are minimal and have zero impact on their performance. Plus, our contribution is secure under the same assumptions taken for the original transaction model of Dusk, called Phoenix.

Second, we introduce \verb!Citadel!: an SSI system fully integrated into Dusk that allows users to acquire licenses (a.k.a. rights), and prove their ownership using ZKPs. By means of our novel and private NFT model, the licenses are privately stored in the Blockchain, and thus, we solve the traceability problem that other solutions had. In particular, we provide a system with the following capabilities:

\begin{itemize}
	\item \textbf{Proof of Ownership:} a user of a service is able to prove ownership of a license that allows them to use such a service.
	\item \textbf{Proof of Validity:} our solution introduces the possibility to revoke licenses. Users can prove ownership of a valid license, that has not been revoked.
 	\item \textbf{Unlinkability:} the SP cannot link any activity of their users with other activities done in the network.
 	\item \textbf{Decentralized Nullification:} our system solves the problem regarding the possibility of reusing the proofs, where a malicious SP could impersonate the user after receiving a valid proof: by means of an on-chain and decentralized nullification, like done in the standard stack of Dusk, proofs cannot be reused.
	\item \textbf{Attribute Blinding:} the user is capable of deciding which information they want to leak to the SP, blinding the value and providing only the desired information.
\end{itemize}

Furthermore, our solution is fully integrated into the Dusk stack, where the deployment of the solution will have minimal impact on other parts already implemented. In this same regard, Dusk has some features allowing users to delegate heavy computing tasks to trusted parties, in a secure and private manner. Our solution has been designed taking all these features into account, and thus, heavy computing tasks of our protocol can be delegated as well. This fact is important, as allows for better scalability and faster integration of our protocol into a wider set of scenarios, like web environments, IoT devices with low computing power, etc.

After describing our solution in full detail, we analyze its security, and finally provide benchmarks using our proof-of-concept implementation\footnote{The proof-of-concept implementation can be found in the following repository: https://github.com/dusk-network/citadel}, to demonstrate its deployment feasibility.

\subsection{Roadmap} 

In Section \ref{sec:preliminaries}, we introduce the preliminaries needed to follow up with the whole paper. In Section \ref{sec:buildingblocks}, we introduce in detail Phoenix, the transaction model of Dusk Network, needed to build our solution. In Section \ref{sec:oursolution}, we describe \verb!Citadel! with full details. We conclude and explain the future work in Section \ref{sec:conclusions}.
